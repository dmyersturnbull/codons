\documentclass[11pt]{nih}

%%%%%%%%%%%%
% Packages %
%%%%%%%%%%%%

% Spacing and layout
%\usepackage[margin=1in]{geometry}
%\usepackage[utf8]{inputenc}
\usepackage{setspace}

% Math
\usepackage{amssymb}
\usepackage{amsfonts}
\usepackage{mathtools}
\usepackage{gensymb}

%% Figures
\usepackage{caption}
\usepackage{subcaption}
\usepackage{boxedminipage}

%% Graphics
\usepackage{graphicx}
\usepackage{rotating}
\usepackage{tikz}
\usepackage{color}

% Tables
\usepackage{longtable}
\usepackage{tabularx}

% Hyperlinks
\usepackage{url}
\usepackage[unicode,hidelinks]{hyperref}

% Bibliography
\usepackage[numbers,sort]{natbib}
 
% Misc
\usepackage{enumerate}
\usepackage{titlesec}

% Chemistry
\usepackage[version=3]{mhchem}



%%%%%%%%%%%%
% Commands %
%%%%%%%%%%%%

% Math
\DeclareMathOperator*{\argmin}{arg\,min}
\DeclareMathOperator*{\argmax}{arg\,max}
\newcommand{\Prob}[1]{\Pr\!\left[\,{#1}\,\right]}
\newcommand{\ProbC}[2]{\Prob{{#1}\,|\,{#2}}}

% Dense enumeration
\newenvironment{denseenum}{
\begin{enumerate}
\setlength{\itemsep}{0.1em}
\setlength{\parskip}{0em}
\setlength{\parsep}{0em}
}{\end{enumerate}}


% TODO notes
\newcommand{\todo}[1]{
\addcontentsline{tdo}{todo}{\protect{#1}}
\textcolor{red}{TODO: #1}
}

% TODO list
\makeatletter \newcommand \listoftodos{\section*{\textcolor{red}{Things to do}} \@starttoc{tdo}}
\newcommand\l@todo[2]
{\par\noindent \textcolor{red}{\textbf{#2}: \parbox{10cm}{#1}}\par} \makeatother


%%%%%%%%%%%%%%
% Formatting %
%%%%%%%%%%%%%%

% Spacing between paragraphs, etc.
\setstretch{1.1}
\setlength{\parskip}{0.4em plus0em minus0em}
\setcounter{secnumdepth}{0}
\setlength{\parindent}{0in}
\setlength{\itemsep}{0in}

%\titleformat*{\section}{\Huge\bfseries}
%\titleformat*{\subsection}{\LARGE\bfseries}
%\titleformat*{\subsubsection}{\large\bfseries}





%%%%%%%%%
% Title %
%%%%%%%%%

\title{\textbf{NIH grant proposal}}
\author{Douglas Myers-Turnbull, Robin Betz, Yunsup Jung}
\date{}

\begin{document}

\maketitle

%\listoftodos

%\tableofcontents


%%%%%%%%%%%%%%%%%
% Specific aims %
%%%%%%%%%%%%%%%%%
\section{Specific aims}

% isoaccepting tRNAs

We propose a large-scale bioinformatics study to identify the effects of synonymous codon usage on protein structure. We intend to address fundamental relationships rather than simple statistical associations by developing a mathematically and statistically rigorous framework, which we will use to address a number of hypotheses. 

\subsection{Aim 1: Develop a comprehensive, verifiable, and rigorous framework to elucidate relationships between codon usage bias and protein structure.}
We will develop a mathematical and computational framework that will allow the robust detection of relationships between codon usage and variables describing protein structure, even when the overall statistical correlation between the two variables is low.

\subsection{Aim 2: Identify particular effects of synonymous mutations on protein structure.}
We will use the framework established in aim 1 to identify proteins whose structures are affected by differential codon usage. We hypothesize that many of these structural changes can cause protein misfolding, which may be clinically important.

\subsection{Aim 3: Elucidate general mechanisms that underlie differential codon usage.}
We intend to use the same framework and pipeline described in the previous aims to investigate our hypotheses that codon usage bias is related to protein domain organization, secondary structure, knotting, folding environment, and structural complexity. Secondarily, we wish to establish codon usage bias as an important biological mechanism.


%%%%%%%%%%%%%%%%
% Significance %
%%%%%%%%%%%%%%%%
\section{Significance}

\subsection{Existing literature}

\subsubsection{Correlation with expression}
It is known that codon usage bias correlates with expression levels \citep{Goetz2005,Gustafsson2004}. There are strong indications that abundance of isoaccepting tRNA molecules is a causal factor of this differential expression \citep{Klumpp2012,Plotkin2011,Najafabadi2007,Tavare1989}. Proteins with high expression levels  contain greater levels of frequently used codons, and frequently used codons are associated with higher levels of corresponding tRNAs. Thus there is believed to be a positive selection for a small, constrained set of codon--tRNA combinations, and concentrations of codons and tRNAs are related in a positive feedback cycle, where bias in either causes a positive selection for bias in the other. Therefore, statistically significant violations of this general trend are interesting because they indicate the presence of other selection biases. Such selection biases may be at the heart of important mechanisms of protein expression, some which are probably currently unknown.

\subsubsection{Terminology}
Because of strong correlation described above, we call infrequently used codons \emph{slow}, and frequently used codons \emph{fast}, except in cases where we address this correlation directly. Furthermore, we consider sequences containing a large proportion of fast codons to have high \emph{codon usage bias}, and sequences containing either a moderate or low proportion to be \emph{unbiased}, even though the bias of sequences with low proportion still deviate from the statistical mean.

\subsubsection{Importance in translational dynamics}
There is also substantial evidence that codon usage bias is also fundamentally linked to translation dynamics  \citep{Bentele2013,Mitarai2013,Cannarozzi2010,Fredrick2010,Marin2008,Buchan2007}. Factors including mRNA secondary structure \citep{Stoletzki2008,Chamary2005,Baim1985}, mRNA stability \citep{Gu2010}, codon--tRNA affinity \cite{Klumpp2012}, and translational errors \cite{Zhou2009,Drummond2008,Najafabadi2007} have been suggested. In addition, studies have shown that codon usage bias is strongly correlated to ribosomal traffic \citep{Cannarozzi2010a,Buchan2007,Komar1999}.  Particularly, was demonstrated that, for efficient translation, overall codon usage should be skewed in favor of fast codons, and there should be a gradient in which slow codons are prevalent at the beginning of the transcript but rare toward the middle and end. \citet{Mitarai2013} used a simple computational model to show that the introduction of even a single slow codon near the end of the transcript can cause ribosomal traffic jams that drastically decrease translation rate, and presumably also expression. This is believed to be due to ribosomal queuing, a ribosome translating an mRNA transcript interacts physically with the translation process of another ribosome on the same transcript. This prevents both ribosomes from proceeding to subsequent codons.

\subsubsection{Effects on protein structure}
Because slow codons can cause pauses in translation, it has been suggested that synonymous codon usage can influence protein folding, leading to different folded states for transcripts with differing synonymous codon usage \cite{Komar2009,Zhang2009,Buchan2007,Crombie1994}.
Several studies have found that codon usage bias is correlated to protein structure \citep{Saunders2010,Biro2006,Adzhubei1996,Gu2003}, including protein secondary structure \citep{Saunders2010,Oresic2003,Gu2003,Thanaraj1996a} and domain organization \citep{Gu2004,Oresic2003}.

%Deana1998 showed regulatory effect
The first known effect of synonymous mutations on protein structure came from \citet{Crombie1994}, who replaced 10 consecutive slow codons with fast codons in the E. coli TRP3 protein. Doing so reduced the native enzymatic activity by 1.5-fold. Still more surprising was the finding by \citet{Sarfaty2007} that a silent polymorphism in the mammalian multi-drug-resistant gene MDR1 altered its folded state and decreased its substrate specificity, without affecting its expression. A recent study by \citet{Zhou2013} found a similar result for a natural mutation in the clinically important circadian rhythm protein FRQ. These studies hint to the existence of more examples of such effects. Furthermore, we argue that additional such cases are likely to be clinically important. 

\subsection{Limitations of existing approaches}

In general, the bioinformatics studies by \citet{Saunders2010,Biro2006,Adzhubei1996,Gu2003} controlled for very few variables and were therefore able to identify only a few clear correlations (most notably with protein secondary structure). However, we hypothesize that these results were negative because the effect of codon usage bias on structure is a weak signal: the effects on most protein structure are minor or nonexistent for most proteins. However, the weakness of the signal does not belie the presence of general mechanisms behind the effects; this is evidence because, as shown by \citet{Zhou2013,Sarfaty2007,Crombie1994}, differential codon usage can dramatically affect the folding of certain proteins. Therefore, we further hypothesize that general mechanisms exist even if few general correlations do, and that controlling for more variables will allow us to elucidate general mechanisms.

\subsection{Further applications}
In addition to clinically significant synonymous mutations and other differential codon usage, our data will have impact on additional applications. It has been recently shown that codon usage bias can be a pivotal factor in de novo protein design \citep{gustafsson2004codon}. Although it is know that designed transcripts should contain mostly fast codons, and that there should be a gradient of codon bias along the transcript sequence, potential unwanted effects of synonymous codon usage on a protein product is not generally considered as part of protein design. We therefore note that the discovery of general mechanisms may be important to this application.

Assessing the impact of codon usage on protein structure has implications for protein folding analyses, especially if correlation is found between fast or slow codons and domain or interdomain regions. Prediction of protein structure given amino acid sequence is one of the foremost problems in biochemistry, however known determinants of structure are few \citep{bryngelson1995funnels} and the predictions made by current computational models frequently fall short of native conformations \citep{das2011four,shell2009blind}. Finding a relationship between codon bias and protein structure would provide considerable additional predictive power to such models.

%%%%%%%%%%%%%%
% Innovation %
%%%%%%%%%%%%%%
\section{Innovation}

The previous studies by \citet{Saunders2010,Biro2006,Adzhubei1996,Gu2003} examining protein structure in the context of codon usage bias have been constrained to examinations of statistical correlations. Although they and many other studies have suggested that differential codon usage may influence protein structure, such effects have only been demonstrated in vivo by \citet{Zhou2013,Sarfaty2007}.

No studies attempting to investigate causal relationships between codon usage bias and protein structure have been thus far published in peer-reviewed journals, and, to our knowledge, no such investigations have been performed. Therefore, we conclude that our proposal is strictly unique in this endeavor

In addition, the previous studies were partly unsuccessful in establishing even statistical correlations. The investigation by \citet{Saunders2010}, which is the most recent, interrogated differences in codon usage  within helices, strands, and coils by applying both Mantel--Haenszel statistics and a $\chi^2$-test. They found few significant differences between the three secondary structural types, but did find a significant decrease in codon usage bias near the transitions between secondary structural elements. They also investigated the hypothesis that slow codons are frequent around domain boundaries; the results in this case were negative. However, these studies relied on simple statistical techniques that lack the power necessary to find statistical associations for weaker signals.

Immense progress has been made in both the quality and quantity of information contained within bioinformatics databases within the past few years, making analyses of codon usage across thousands or more proteins possible. Examining larger sample sizes enables the study of family or activity based codon bias as well as allowing for a more robust statistical analysis when compared with previous studies where fewer than 300 proteins were examined.

%%%%%%%%%%%%
% Approach %
%%%%%%%%%%%%
\section{Approach}

%%%%%%%%%%%%%%%%%%%%
% Preliminary data %
%%%%%%%%%%%%%%%%%%%%
%\subsection{Preliminary data}
% Justify our hypotheses

%%%%%%%%%
% Aim 1 %
%%%%%%%%%
\subsection{Aim 1}

\subsubsection{Determination of causation}
Ultimately, we are interested in causation related to codon usage bias. The analyses described above are independently useful, both for novel findings and verification of previous results. While we argue that such results are interesting, biologically relevant, and warrant further investigation, they may indicate only indirect correlations. It is entirely possible for one variable to account for another in whole or in part; for example, domain boundaries may be enriched for slow codons only because they contain more strands than helices, or that solvent accessibility around domain boundaries may be the more fundamental explanation.


\subsection{Data sets}

We primarily aim to investigate the relationship between secondary structural elements (SSEs) and slow codons, with the aim of finding biases towards slow codon enrichment in certain elements. The null hypothesis for this situation is that there is no meaningful relationship between frequency of slow codons and domain features. However, we hypothesize that such a relationship does exist, as previous research indicates the introduction or removal of slow codons has a dramatic effect on protein expression.

We will construct 5--10 species-dependent data sets, covering eukaryotes, invertebrates, and Homo sapiens. Using more than 1 data set will allow us to control for inter-species variation. We will necessarily limit the data sets to contain only genes annotated with structures in the Protein Data Bank (PDB) \citep{pdb}.

The content of databases that permit public deposition of entries, such as the NCBI and the PDB, are significantly biased toward sequences and structures that are of experimental interest. To control for such bias, we will cluster the genes by alignments using Basic Local Alignment Tool (BLAST) pairwise, then clustering by 40\% sequence identity to remove homologs.

Coding mRNA sequences for the protein will be obtained from the NCBI Reference Sequence Database (RefSeq) \citep{refseq}. Sequences will be correlated to structures by means of Structure Integration with Function, Taxonomy, and Sequence (SIFTS) database \citep{sifts}.

\subsection{Quantification of bias}
Bias towards fast or slow codons in mRNA sequence will be established using the codon adaptation index (CAI) described by \citet{sharp1987codon}. Codons are assigned weights depending on the frequency of their corresponding tRNA in the appropriate organism:
\begin{equation}
w_{ij} = \frac{X_{ij}}{X_{i^*}}
\end{equation}

The codon usage bias in a particular region is then the geometric mean of the codons that comprise it:
\begin{align}
CAI &= (\displaystyle \prod_{k=1}^L w_k )^{\frac{1}{L}} \\
&= \exp(\frac{1}{L} \displaystyle \sum_{k=1}^L \log w_k)
\end{align}


\subsubsection{Hypotheses}
We are currently interested in several hypotheses and open questions:
\begin{enumerate}
\item \textbf{Secondary structure}
Are particular secondary structural elements (SSEs) enriched for slow codons? Are the transitions between SSEs, in agreement with \citet{Saunders2010}, enriched for slow codons?

To investigate this hypothesis, we will use Define Secondary Structure of Proteins (DSSP) \citep{Joosten2011,Kabsch1983} to identify SSEs. Although DSSP is currently outperformed in accuracy by sophisticated machine learning methods, the use of machine learning algorithms such as PROTEUS presents significant pitfalls in performing statistical analysis because decisions by the classifier depend on the training data, and individual classification decisions are unstable subject to arbitrary bounds. Because our methodology depends so heavily on such analysis to improve overall sensitivity, we consider this loss of statistical rigor unacceptable.

We will first attempt to verify the results of \citet{Saunders2010} by duplicating their approach. Specifically, we will calculate the distribution of codon usage bias within helices, strands, and coils.We will apply a Mantel--Haenszel and a $\chi^2$-test to determine significance between the categories. Secondly, we select 2--5 large data sets (at least 2,000 genes each), and expand the categories to also include, at a minimum, $3_{10}$-helices, $\beta$-helices, and $\pi$-helices. We will also distinguish between $\beta$-sheets and isolated $\beta$-strands. We will then determine significance using the same statistical techniques.

To address the second part of this question, we will calculate codon usage bias around the transitions between SSEs, using windows of 0, 1, 2, 4, and 6 residues.

\textbf{Preliminary data:} The research by \citet{Saunders2010,Oresic2003,Gu2003,Thanaraj1996a} all found statistically significant correlations between codon usage bias and protein secondary structure.
%we  investigated the hypothesis that strands are enriched for slow codons on a novel data set of 437 S. cerevisiae genes annotated by structures in the PDB. The results showed that \todo{something}.

\item \textbf{Domain boundaries}
Are domain boundaries enriched for slow codons?
To investigate this question, we will use domain classification by the Structural Classification of Proteins version 2 (SCOP2) \citep{Andreeva2013,scop}. As a manually curated database, SCOP2 is very reliable for domain assignment. Although we admit that its coverage is limited (167,547 domains in 59,514 protein structures, about $62\%$ of the PDB), we argue that will still result in sufficiently large data sets.

Using the intersection of SCOP and the data sets described above, we will calculate codon usage bias near and apart from domain boundaries, using windows of 0, 2, 5, 8, and 12 residues from the domain boundary positions. We will then apply a Mantel--Haenszel test to determine significance.

\textbf{Preliminary data:} The research by \citet{Gu2004,Oresic2003} showed an enrichment, though \citet{Saunders2010} found no statistically significant difference. Because all three studies used different methods, the work by Saunders et al. did not supersede the results by Gu et al. and Oresic et al; rather, there is still evidence that a significant correlation exists.

\item \textbf{Structural complexity and knotting}

Due to the complexity of the energy funnels for the folding of many proteins \citep{bryngelson1995funnels,Onuchic2004}, we hypothesize that proteins with complex tertiary structures are enriched for slow codons. In particular, we hypothesize that residues with more interactions with other residues in a protein are more likely to be encoded for by slow codons, and that proteins with more self-interactions---that is, have more interactions to overcome during the folding process---are enriched for slow codons.

We also conjecture that knotted proteins are enriched for slow codons, and that codon usage bias is inversely correlated with the complexity of the knot. To investigate these hypotheses, we will identify knots using the existing algorithms by \citet{Lai2012,Virnau2006}. Although neither algorithm can detect all knots, we argue that this is sufficient for our purposes.

Knots are classified up to isomorphism by the Jones polynomial \cite{Jones2005}, a knot invariant that revolutionized knot theory in part because it is easy to compute. It is defined in terms of an arbitrary projection $D$ of a knot $K$ by:\\
\begin{equation}
f_D(A) = (-A^3)^{-w(D)}\langle D \rangle
\end{equation}
where $w(D)$ is the writhe of the diagram $D$, and $\langle D\rangle$ is its Kauffman bracket.

The result of the computation is a Laurent polynomial that uniquely identifies the knot. Moreover, the number of terms in the Laurent polynomial and its coefficients can be used to define a complexity $\xi(K)$ of the knot. We hypothesize that the knot complexity $\xi$ is associated with the CAI of a gene.

\textbf{Preliminary data:} The pipeline used in this determination is available at \url{https://github.com/dmyersturnbull/codons}. Using the same data set of 437 S. cerevisiae genes, we investigated the correlation between codon usage bias and two simple measures of structural complexity:
\begin{enumerate}[a)]
\item Sequence length
\item The average squared distance between all residues:\\
\begin{equation}
\displaystyle \frac{1}{n^2} \sqrt{\displaystyle \sum_{R \in A} \sum_{S \in A} ||R - S||_2^2 }
\end{equation}
where $A$ is a single protein, and $||R - S||_2$ is the two-norm of the difference between the vectors corresponding to residues $R$ and $S$.
\end{enumerate}

In this preliminary analysis, we found no statistically significant correlations for either measure using Pearson's correlation coefficients ($r = 0.19$ for (a), and $r = 0.117$ for (b)). However, even this data should be further analyzed, as Pearson's correlation assumes a normal distribution and responds only to linear correlations.

In the case of structural complexity, we are interested in statistical associations that are potentially nonlinear. For this purpose we will use the Hilbert--Schmidt Independence Criterion (HSIC) \cite{Gretton2005} to identify statistical associations, both linear and nonlinear.

The HSIC is a statistical test defined using a generalization of the Frobenius norm for linear operators called

The Hilbert--Schmidt (HS) norm is an operator norm defined for an arbitrary operator $C: G \rightarrow F$:

: for an operator $C: G \rightarrow F$, where $G$ and $F$ are required only to be orthonormal bases of separable Hilbert spaces.
\begin{equation}
||C||_{HS}^2 = \displaystyle \sum_{i,j} \langle C \nu_i, \mu_j \rangle_F^2
\end{equation}
where $\nu$ and $\mu$ are orthonormal bases of $F$ and $G$. The HS norm is extremely similar to the Frobenius norm for matrices. However, the generalization permits its use for arbitrary operators rather than only linear operators (specifically, the only requisite is that $F$ and $G$ are Hilbert spaces).
A cross-covariance operator $C_{xy}$ is then defined, and the HSIC is defined as:
\begin{equation}
HSIC(p_{xy}, F, G) := ||C_{x,y}||_{HS}^2
\end{equation}
where $p$ is a probability measure.
\cite{Song2012}
The HSIC is a measure of statistical dependence: the higher the HSIC between $F$ and $G$, the greater the dependence between $F$ and $G$. Moreover, the HSIC has been proved to be a measure of arbitrary statistical dependence \citet{Song2012a}. Thus the advantage of this method is that it permits the determination of nonlinear associations without the use of a kernel trick or artificial regularization. This should allow us to identify potential relationships between protein complexity and codon usage bias in a manner that is sensitive and statistically robust, and that does not depend on an arbitrary selection of kernels or regularization terms.

The approach of \citet{Song2012a} for feature selection and optimization of the HSIC was shown to have good results, and we will abide by this procedure.

%\item \textbf{Solvent accessibility and oligomeric assembly}

%\cite{bryngelson1995funnels}


\end{enumerate}

%The determination of causal relationships from correlation data consists of two tasks:
%\begin{enumerate}
%\item The removal of indirect correlations, and
%\item The assignment of a directions to each relationship
%\end{enumerate}

%To address the first task, correlations will be analyzed again while controlling for the effects of other variables. Statistical association will be measured using the Hilbert--Schmidt norm \cite{Song2012}

%Directionality will be determined manually. \todo{Uh. What?}

%To formulate these two tasks concretely, we will model the entire problem as a Markov Random Field.



%\subsubsection{Mathematical model}

\subsubsection{Analysis of accuracy}
Many proteins will be examined over a number of species. While our preliminary analysis has been performed in approximately 500 yeast proteins, we hope to analyze over 10,000 human proteins in hopes of maximizing sample size.

Codon usage bias will be examined on domain boundary regions and non-boundary regions, and the overall CAI value will be compared across these regions. If a significant difference in CAI across secondary structure features is found, the hypothesis will be supported.
\cite{Liu:2011p8245}


%%%%%%%%%
% Aim 2 %
%%%%%%%%%
\subsection{Aim 2}
We also aim to examine existing proteins that have been demonstrated to be affected by the removal of slow codons, and characterize in what circumstances a protein could be reasonably assumed to require slow codons for normal function.

We will run our analysis on species-independent protein families that contain at least one protein shown experimentally to require slow codons. We will complement our structural analysis with a phylogenetic examination of conserved regions to see if the bias towards slow codons is strictly evolutionarily enforced or if it is an isolated development in some species.

%%%%%%%%%
% Aim 3 %
%%%%%%%%%
\subsection{Aim 3}

The approach for this aim is largely documented in the above section for Aim 1, as the former will follow from the latter. However, as an auxiliary analysis, we will investigate the effects of synonymous mutations in detail for select cases using MD-based folding software. These cases will be selected according to the success or failure of our hypotheses in Aim 1. Although de novo folding is still largely unsolved, and de novo folding algorithms are still in their infancy, such an investigation may still reveal insight for some cases that is unavailable through other means \cite{Zhang:2008p3335}.



%%%%%%%%%%%%%%%%
% Bibliography %
%%%%%%%%%%%%%%%%

\bibliographystyle{plainnat} 
\bibliography{refs}
\appendix


\end{document}


